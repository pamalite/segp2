\documentclass[10pt, a4]{article}
\usepackage{times}

\begin{document}

\title{Software Engineering Group Project - 7096A}
\author{Semester 1 Week 10 Meeting - Sub-Group 1}
\date{Wed May 21, 2008 (1400hr)}

\maketitle 

\noindent {\large \textbf{\underline{Configuration and Documentation of Project Repository (GitHub)}}}\\
\noindent {\normalsize \textbf{Assignee:} Alexander Egan}\\
\noindent {\normalsize \textbf{Hours:} 21hrs (approx)}\\
\begin{enumerate}
\item {Task Description}
  \begin{itemize}
  \item Configure and Document the Processes and Procedures for Updating the GitHub Repository
  \end{itemize}
\item {Status}
  \begin{itemize}
  \item I edited the repository usage document. This was to make it more detailed about using git and the repository structure in general. Many group members are still not comfortable using git and the system we have decided upon so hopefully this should fix that. A testing process document has also been written. This details how and when testing should be performed. It also references the repository usage document and explains what testing should be done and who is responsible for it at each point of code escalation in the repository structure. This document should allow for better development through a focussed testing process. Code for milestone 1 was combined and the test cases were run over it. As most of the changes made were display changes, these were tested manually as an automated solution could not be found at this stage. The existing test cases passed. 
  \end{itemize}
\end{enumerate}

\[\emph{End}\]

\end{document}





