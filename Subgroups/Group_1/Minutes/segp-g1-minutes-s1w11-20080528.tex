\documentclass[10pt, a4]{article}
\usepackage{times}
\usepackage{a4wide}
\begin{document}

\title{Software Engineering Group Project 4001/7096A}
\author{Semester 1 Week 12 Meeting - Team 1}
\date{Wed Jun 4, 2008 (1415hr)}

\maketitle
 
\noindent {\large \textbf{\underline{Minutes of Meeting on Wed May 28, 2008}}}\\
 
\noindent \textbf{Attendees:} Dr. David Hemer, Dr. Li Jiang, Alex Egan, 
Filimoni Lutunaika, George Sainsbury, Xiaodong Cui\\

\section{Milestone 2 Reviews}

\subsection{Ticket 146 - Continuous Running of Daemon Test Script}

\paragraph{}Fil updated the group on his progress with the continuous 
testing of the Earth daemon as raised in Ticket 146. He had completed
preliminary testing using both his own and the uni machine with satisfactory 
results. The test script verifies the performance of the daemon by 
manipulating the test directory on the local machine and checks that 
these changes are captured by the daemon in a timely fashion and updated 
on the PostgreSQL database. This manipulation of test directory involves 
creating, deleting, moving and timestamping (touching) files and folders at 
random. He had also included a text file which contains specific 
instructions on how to configure and execute the script in a test environment.

\paragraph{} David responded to this by raising the issue of how these 
progresses are being recorded. Ideally, whatever modifications and changes 
that had been made for this task should be updated on RSP's Trac system
for the benefit of other interested developers. Even if the source codes 
have not been pushed onto the RSP repository, information about the status
of each task or ticket should still be updated together with the relevant 
reference to the SEGP2 repository.


\subsection{Ticket 138 - Query Optimisation under Radial View}
 
\paragraph{} Cui informed the group about his progress on the
optimisation of queries when filtering under radial view. He had found 
that a significant part of the delay to the current implementation is 
not due to the actual execution of the query on the database. He also
deduced that the delay was somewhere on the Rails side of the application 
after finding out that running the raw query on the PostgreSQL database 
only took a small fraction of the measured turnaround time for the same query. 
Nevertheless, he had made some modifications to the current query which 
produced a slight improvement to the total running time of the query.

\paragraph{} David discussed the group's views on how to proceed with 
this particular ticket and mentioned that it was alright to set this 
Ticket aside for the time being and have its priority downgraded. 
This way, the group would be able to take on a new ticket of high 
priority that has application-wide improvements for the next milestone.


\subsection{Documentation of Project Repository and Testing Process}

\paragraph{} Alex discussed the outcome of presenting the \emph{Project 
Repository Document} and \emph{Testing Process Document} to the main 
project group of Monday. The main group had subsequently decided to 
adopt the procedures specified in these documents and reorganised the
hierarchy of the physical repository accordingly.


\section{Milestone 3 Planning}

\paragraph{} David sought clarification on how the group had come up with 
the tasks (tickets) for Milestone 3. Fil mentioned that the tasks (tickets) 
were selected based on how they would complement what had been the tasks
undertaken during the development sprint for Milestone 2. Furthermore, 
since the other two subgroups had decided to concentrate on plugins 
during the current development sprint, Group 1 decided to investigate
a few other high priority tickets instead.

\paragraph{} George informed the group that he would be investigating the 
process of creating a gem for Earth. If the actual process simply involves 
pointing all the constituent components onto a target build directory then 
it could be a simple matter of identifying all the relevant dependencies
and creating the Earth gem. David then added that the future integration 
of plugins should also be considered when undertaking this task.

\paragraph{} Alex informed the group about his plans to investigate and 
implement the remove feature for the Earth daemon. He highlighted the current 
shortcomings of the Earth application due to the lacking remove feature. 
In response, David highlighted the ways in which this task could be 
attempted such as whether to simply flag the records under the removed folder 
in the file hierarchy or permanently delete all the related records 
from the database. He also advised that some further analysis of these 
issues be considered in order to find the best way to approach the task. 
Nevertheless, he recommended trying out the most simple solution first 
before fully committing to a more complex alternative.

\paragraph{} David then raised some concerns on how the time estimates 
in the Group 1 Milestone 3 plan were obtained. Fil admitted that the group 
did not reach this level of detail while selecting the task, instead he simply 
deduced the time that each member would be expected to spend on the project 
during this development sprint. David then reiterated the need to do proper 
planning particularly when it concerns matching development tasks to the
resources at hand. The group might have to consider some preliminary scoping 
of each task in order to obtain a better appreciation of each task and a more
accurate time estimate.

\paragraph{} David also commented on the significant portion of development 
time that the group has decided to allocate for integration testing phase. 
Fil explained that based on the group's rather mediocre experiences during 
the previous milestone, it had been decided that more resources (time) be 
expended at the integration testing phase. These additional timing constraints
would effectively reduce the actual development period. However, the strategy
would allow for a better code review strategy than would be otherwise possible.

\paragraph{} George then suggested that the next meeting be scheduled at the 
same time (1415hrs) and same day (Wed) next week. This was agreeable 
to everyone and the meeting proper ended at around 1500hrs.


\section{Other Matters}

\textbf{Next Meeting:} Jun 4, 2008 (Wed) @ 1415hrs\\

\[\emph{End}\]
 
\end{document} 
