\documentclass[oneside, 10pt, a4]{article}
\usepackage{times}

\begin{document}

\title{Software Engineering Group Project - 7096A}
\author{Semester 1 Week 11 - Group 1}
\date{Mon May 26, 2008}

\maketitle

\noindent \textbf{Members} Alex Egan, Filimoni Lutunaika, George Sainsbury, Xiaodong Cui\\

\section{Introduction}

\paragraph{}
Group 1 was assigned Tickets 138 and 146 for the Milestone 2 development leg. 
Ticket 138 highlighted the degradation of the performance of the Earth Project 
application when filtering search results while in Radial View. Ticket 146 raised
 the ineffectiveness of the existing integration testing script amidst the expanding 
 features and functionalities of the Eart Project application. In addition , Group 1 was
  also tasked with conducting the integration testing of the code deliverables from Milestone 
  1 and establishing procedures on the effective use of Group's various GitHub repositories.

\paragraph{}

\section{Resources}
Group 1 comprised an equal split of 2 Bachelor of Engineering (Software Engineering)
 students and 2 Master of Software of Engineering students. This provided the Group with
  approximately 56 hours of available development time per week. This estimate is based on 
  the 8 hours per week coursework commitment that the school expects from each BESE student 
  and 20 hours per week for each MSE students.

\paragraph{}

\section{Task Allocation}

\paragraph{}
For Milestone 2, the distribution tasks amongst the members were based on among other 
things the hours that each member was expected to commit to the project. As a result, the
 major Group 1 tasks of investigating Tickets 138 and 146  were assigned to the MSE students Cui 
 (138) and Filimoni (146). The other relatively light but still important tasks were assigned 
 to the BESE students Alex (GitHub procedures) and George (Milestone 1 Solution Integration). 
 This particular way of distributing the tasks amongst the Group 1 members was considered fair 
 during the planning stage of the Milestone 2.

\section{Activity Summary}

\paragraph{}
While each member seemed to have successfully made a serious attempt in their respective 
tasks, the clear isolation of each Group 1 task prevented any significant collaborative
development opportunities within the Group. These collaborative efforts generally provide 
the added support to properly and thoroughly complete any software development task. However, 
based on the Group's Milestone 2 achievements, it could be safely assumed that the distribution 
of tasks was at least fair. (Please refer to the accompanying reports for a detailed and complete 
description of each development task undertaken by Group 1)

\section{Improvements}

\paragraph{}
A further improvement that the Group could adopt as part of its task distribution 
strategy is to assign each major development task to more than one member. This would 
help ensure the continuity of effort on each task and be less reliant on the individual 
effort of each member. This planned strategy would ensure that each significant design 
and development issue would be seriously considered by more than one group member. The 
collaborative efforts should produce better outcomes in terms of quality design decisions 
and robust implementations.

\section{Challenges}

\paragraph{}
Given that Milestone 1 was practical orientation for every Group member to the Earth Project, 
Milestone 2 provided the first real test for each Group 1 member. Based on the Group's performance 
during the Milestone 2 sprint, there were no significant deficiency detected in terms of 
communication or development skills. Despite being under resourced (compared the other groups), 
the Group had managed to produce comparable results. This certainly bodes well for the Group\'s 
performance and with the valuable experiences from Milestones 1 and 2, the Group could anticipate the 
same if not better results for Milestone 3.

\newpage

\section{Skill Set Analysis}

\paragraph{}
During the course of this Project, the Group would be aiming to expose all its members 
to the various types of software development tasks that the Earth Project provides. 
In order to accomplish this, the Group had drawn up a skill set matrix wherein each members\' 
assigned tasks in each Milestone is assessed and classified accordingly. This would certainly 
enhance each Group member's skill set through direct exposure and ensure the successful 
attainment of Group's development tasks in later milestones.

\paragraph{}
The following is a snapshot of the matrix at the completion of Milestone 2.

\begin{table}[ht!]
\begin{tabular}{|p{5cm}|p{1cm}|p{1cm}|p{1cm}|p{1cm}|} 
\hline
 & & & & \\
Skill Set & AE & FL & GS & XC\\
 & & & & \\
\hline
 & & & & \\
Db Admin & & & & +\\
 & & & & \\
\hline
 & & & & \\
Plugins & & & & \\
 & & & & \\
\hline
 & & & & \\
Documentation & + & + & & \\
 & & & & \\
\hline
 & & & & \\
Testing & & + & + & \\
 & & & & \\
\hline
 & & & & \\
Configuration Management & + & & &\\
 & & & & \\
\hline
 & & & & \\
Interface Design & & & & \\
 & & & & \\
\hline
\end{tabular}
\end{table}

\section{Conclusions}

\paragraph{}
Group 1 managed to do further investigation on both Tickets 138 and 146 and the 
relevant codes could be obtained from the GitHub repository. The integration testing of 
Milestone 1 source codes was successfully carried out and posted on the GitHub repository.
 The relevant GitHub procedure manuals were also produced and made available to all Group members 
 through the GitHub repository.

\[\emph{End}\]
\end{document}