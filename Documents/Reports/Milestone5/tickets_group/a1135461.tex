\documentclass{article}
\begin{document}
\markright{JONATHAN VELASCO (1135461)}
\pagestyle{headings}

\begin{center}
{\LARGE\textbf{\underline{{Individual Milestone 5 Report}}}}
\end{center}

\section*{Executive Summary}

% I took responsibility for the administrative tasks and elaboration of agendas and minutes.  Also, I was originally, together with the rest of the group, assigned to work extracting the possible API methods, however, this task was then taken by Ken and Jian.  And I was suggested by Ken to start studying the GUI together with Sahil.

\section*{Tasks and Activities Performed}

\begin{itemize}
	\item Re-installation of software
	     \begin{itemize}
                \item Description: Due to a power failure in my laptop (equipment I dedicated for earth) I had to install/update  the needed software on my PC (I had already some previous version installed but they needed to be updated to be able to work with earth).  However, when trying to update the rubygems I was getting an error related to rdoc.  After searching on the internet the problem seems to be related to the installation of rdoc.  After several attempts to fix it I decided that it was faster to just make a new partition and install all the needed software from scratch following the step-by-step guide I putted together previously
                \item Idea/Solution: new partition on disk
                \item Affected files: N/A
                \item Git commits: N/A
                \item Estimated time taken (planned): N/A (was not planified.  Though I could fix it in 0.5 hours)
                \item Estimated time taken (actual): 2 hours (approx 0.75 trying to fix the rdoc error, and 1.25 doing the actual diskpartition plus installation of ruby, rubygem, postgres and the other libraries needed for earth)
            
	     \end{itemize}
	\item Integration of previous millestone
	     \begin{itemize}
	        \item Description: Start looking a what should be integrated from the 3 groups from the previous millestone.  I then studied the instructions elaborated by Alex and Mohammad and also followed the link from the source they gave.  However, after several attempts, following different methods and ideas I discovered I was not able to perform the integration due to restrictions on my privileges. .This issue was then reported to Mohammad  (who corroborated me that the problem was the privileges) in order to grant me the necessary privileges in order to be able to the integration for the next milestone
		\item Idea/Solution: report privilege issue to Mohammad
		\item Affected files: N/A
		\item Git commits: N/A
		\item Estimated time taken (planned): 8 hours
		\item Estimated time taken (actual): 2 (due to privileges restrictions) hours
	      \end{itemize}
	\item Due to the technical impediments to perform the integration, I then join Ming and Callum in order to help them with the image task.   We conducted a research about how this could be done.  We then found a piece of code which we modified to accommodate to out necessity.
            \begin{itemize}
                \item sub-task 1
                \begin{itemize}
                    \item Description: Research on how to extract information from the images
                    \item Idea/Solution: found code on internet which could be adapted to our needs
                    \item Affected files:N/A
                    \item Git commits: N/A
                    \item Estimated time taken (planned): 1 hour
                    \item Estimated time taken (actual): 2.5 hour
                \end{itemize}
                \item sub-task 2
                \begin{itemize}
                    \item Description: Study and modifications were done to the code found.  
                    \begin{itemize}
                     \item Changed it from looping through all files from the current directory and extracting their information, to only extract information form the file specified on a given variable.
                     \item Extraction the piece of code that allowed to get the file name from I/O (not needed)
                     \item Also stopped from processing all files (even if they are not images) to only process images file.
                    \end{itemize}
                    As this work was assigned to Callum and Ming, they already had a solution in mind which involved the files and tables that their group did in previous milestone; as I do not have access to it, I could not test the code in earth, therefore, all the changes where sent to Callum and Ming.
                    \item Idea/Solution: Study and adaptation of code to meet our necessities
                    \item Affected files: image\_size.rb (original code found)
                    \item Git commits: N/A (changes where mailed, not need of GIT for it)
                    \item Estimated time taken (planned): 2 hour
                    \item Estimated time taken (actual): 3.5 hour
                \end{itemize}
            \end{itemize}
       
                
	\item Elaboration of documents
	     \begin{itemize}
	         \item Description: Elaboration of report (milestone 4)
	         \item Idea/Solution: N/A
	         \item Affected files: \texttt{Documents/Reports/Milestone4/Group\_3/a1135461.1.tex, Documents/Reports/Milestone4/Group\_3/a1135461.tex}.  
	         \item Git commits: f7e675bda62c13ab2c744572d397d85fe979de7a
		 \item Estimated time taken (planned): N/A
	         \item Estimated time taken (actual): 0.5 hours 
	     \end{itemize}
\end{itemize}

\section*{Resource Contributions}

In addition to all the time specified above, there where several meetings realted to mileston 5 and the new groups.  Some of these meetings were held before the reform of the groups; on such meeting,  Ken, Jian and I disccussed about this possibility of reforming the groups and evaluate different options on how to make the new groups.  The total time spent on meetings was approx 5 hours, given a total for milestone 5 of 15.5 hours.


\end{document}  