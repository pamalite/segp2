\begin{center}
{\large\textbf{\underline{{Filimoni's Milestone 5 Report}}}}
\end{center}

\section*{Executive Summary}

For the current milestone (M5), I was reassigned to the Plugin Focus Group. This group comprised all the seven (7) MSE students and 
was tasked with investigating the plugin system for the Earth project. The investigation showed that while the file monitor class 
(file\_monitor.rb) had been designed as a plugin, the actual implementation was tightly coupled to the underlying earth daemon code.
This would make it difficult to expand the features and functions of the Earth project. So the investigation also involved modifying
the Earth daemon code to make it modular and provide a standardised plugin interface.


\section*{Tasks and Activities Assigned}

\begin{itemize}
	\item Examine and Verify Operations of Original Daemon Design and Plugin Support.
		\begin{itemize}
			\item Sub-task 1: Locate and learn about current group progress with the plugin system.
	      	\begin{itemize}
					\item Description: This involves reading the documentation (investigation report) of the plugin system by Group 3 in previous milestones.
					\item Idea/Solution: The existing documents revealed that the original code incorporates the option to implement the file monitor class as a plugin.
					\item Affected files: \texttt{Plugin\_Design\_Notes.pdf}
					\item Git commits: N/A.
					\item Estimated time taken (planned): 10 hours
					\item Estimated time taken (actual): 16 hours
				\end{itemize}
			\item Sub-task 2: Identify all plugin-related designs in codes.
			   \begin{itemize}
					\item Description: While the relevant plugin notes (Sub-task 1) showed the design concepts and explained its implementation, it did not show where these design decisions are implemented in the source codes. As a result, this Sub-task (2) involved identifying the relevant source code files and code snippets in order to attain a better understanding of the original plugin design and implementation. 
					\item Idea/Solution: N/A. 
					\item Affected files: \texttt{lib/earth\_plugins/file\_monitor.rb}, \texttt{script/earthd}
					\item Git commits: \texttt{(ssurfer/earth) 0f0464179d5da4d0d22b3e9345bece9aa02136e4}
					\item Estimated time taken (planned): 20 hours
					\item Estimated time taken (actual): 18 hours
				\end{itemize}
		\end{itemize}
	\item Eliminate the Tight-Coupling of Earth Daemon and File Monitor Plugin
		\begin{itemize}
	   	\item Sub-task 1: Remove hard-coded elements of the file monitor plugin from daemon code.
	      	\begin{itemize}
	         	\item Description: The original form of the Earth daemon involves instantiating the file monitor class and subsequently invoking the (\@file\_monitor) object methods to update the database records of monitored directories. 
					\item Idea/Solution: This particular Sub-task involved changing this process into a generic approach where the daemon simply invokes a general plugin method (main) that is specified in the EarthPlugin superclass and implemented through method override in all the respective plugin subclasses.
					\item Affected files: \texttt{lib/earth\_plugins/file\_monitor.rb}, \texttt{script/earthd}
					\item Git commits: \texttt{(ssurfer/earth) 0f0464179d5da4d0d22b3e9345bece9aa02136e4}
					\item Estimated time taken (planned): 30 hours
					\item Estimated time taken (actual): 24 hours
				\end{itemize}
			\item Sub-task 2: Verify operation of modified daemon and file monitor plugin codes after de-coupling.
	      	\begin{itemize}
					\item Description: Verification that de-coupling of the daemon and the file monitor plugin works as planned.
					\item Idea/Solution:  This Sub-task involved ensuring that in addition to the preservation of the original behaviour of the Earth daemon and file monitor codes implementation, no unnecessary performance degradation occurred as a result of the design modification.  
					\item Affected files: \texttt{lib/earth\_plugins/file\_monitor.rb}, \texttt{script/earthd}
					\item Git commits: \texttt{(ssurfer/earth) 0f0464179d5da4d0d22b3e9345bece9aa02136e4}
					\item Estimated time taken (planned): 10 hour
					\item Estimated time taken (actual): 16 hours
				\end{itemize}
		\end{itemize}
	\item Identify Effective Plugin Design
		\begin{itemize}
	   	\item Description: Examined ways to generically implement or load plugins from the daemon. 
	      \item Idea/Solution: The most straight-forward way to implement the plugins was to simply load all existing plugins from the database. Conceptually, this idea also implies that the plugin itself has to have extra-functional features and logic blocks that determines runtime invocation and/or deactivation. The downside of this simplistic implementation means a signficant overhead on the plugin class, which would probably discourage the development of additional feature plugins. A further downside is that any interested plugin developer would need to access and study the daemon code in order to determine how his/her plugin will 'hookup' onto the daemon.
	      \item Affected files: \texttt{lib/earth\_plugins/file\_monitor.rb}, \texttt{script/earthd}
	      \item Git commits: \texttt{(ssurfer/earth) 0f0464179d5da4d0d22b3e9345bece9aa02136e4}
	      \item Estimated time taken (planned): 20 hours
	      \item Estimated time take (actual): 12 hours
		\end{itemize}
\end{itemize}

\section*{Resource Contributions}

After contributing about 86 hours over the course of 3 weeks during Milestone 5, I believe that more progress could have been achieved
if the organisation of existing documents about the plugin system had been up-to-date and less cryptic than what it was at the commencement
date of this current development sprint(M5). As a result, a significant portion of this 86 hours was spent on verifying the accuracy and
correctness of the documentation. \\

However, I believe that this situation was only applicable to those of us who were not part of Subgroup 3 during the previous development sprints.
Subgroup 1 (my former group) had been exposed to every aspect of the Earth project except the plugin system in previous sprints. This might help explain
why a few of us were not as well-versed with the available documentation at the beginning of this sprint. This situation was rectified with
the appropriate updates of the plugin system documentation and a special briefing session conducted by Ken.


\section*{Room for Improvement}

\begin{itemize}
   \item Better planning: Having coarse-grained Sub-tasks provides room for unproductive investigations that resulted in wasted resources.
      \begin{itemize}
         \item Plan: Refine subtasks further into 3-4 hours tasks for better monitoring and improved development focus. 
      \end{itemize}
\end{itemize}
