\documentclass{article}
\usepackage{a4wide}
\usepackage{hyperref}

\begin{document}
\markright{SEGP2 Plugin Group (13$^{th}$ October, 2008)}
\pagestyle{headings}

\begin{center}
{\textbf{\underline{SEGP2 Plugin Group Milestone 6 Report}}}
\end{center}

\noindent{\textbf{Total Hours Spend:} approx. ??? hours}

\section*{Executive Summary}

Milestone 6 began by reviewing what were the leftover work and issues needed to be addressed. The following were the tasks that had been set by the group at the beginning of the sprint:

\begin{itemize}
    \item Finish up the 3$^{rd}$ pillar of plugins: deployment.
    \item Create plugins and apply what was created and set up plugins development into them.
    \item Front-end plugins union: Finalize and have the plugins framework set.
\end{itemize}

The first and second items were taken up by Ida, Mohammad, Fil and Ken, while Xiaodong, Keane and Huang Jian were assigned to the last item. 

Ida managed to update the \texttt{plugins\_manager.rb} to support extension point, introduced by Mohammad, during the installation of a daemon plugin. She also managed to add an uninstall method and accessor script to uninstall the a plugin from Earth. Mohammad provided his expertise to assist Ida in getting the implementation correct. Ida had spend ?? hours on updating the scripts, while Mohammad spend ?? hours to lend his expertise to her, while implementing the file metadata as an extension point. 

Ken created the \texttt{list\_plugins} script to enable the administrator to see what plugins had been installed. Also, he unified all the plugins related scripts (\texttt{sign\_plugin}, \texttt{install\_plugin} and \texttt{uninstall\_plugin}) into \texttt{earth\_plugins}. This root accessor script will call the appropriate script to perform the required plugin task. Ken had spend 4 hours creating the scripts. He also spend 3 hours to help in preliminary integration tasks.

Fil created a daemon plugin that redirects the daemon status to the front-end. He spend ?? hours on it.

Xiaodong, Keane and Huang Jian had conceptualized and implemented their versions of GUI plugins framework into functional prototypes.

Xiaodong and Keane's implementation used the Rails plugin generation mechanism to get the plugins installed. They require the user to add specific codes into the controllers to activate the plugins. On the other hand, Huang Jian's implementation used his own way to get the plugins installed, without any code activation in the controller. Both ways are similar, but the difference is only in the ways the framework is implemented. Thus, they have agreed to work togather further to consolidate the ideas, or finalize one one as the final framework, while the other as a proof-of-concept branch. 

All three spend ?? hours in bringing the concepts to functional implementations. 
 
\section*{Individual Reports}

Attached are the individual reports of each member of the group. 

\begin{center}
{\LARGE\textbf{\underline{{Ken's Milestone 7 Report}}}}
\end{center}

\section*{Executive Summary}

I helped in strategizing how integration should be done, togather with Fil, Mohammad and Jon. I suggested to use branches instead of forks to have the integration done. In the end a new fork was created and in that fork, there were several branches created as a buffer for each integration stages. The stages are partial integration between groups. The first two groups were first integrated, then followed by plugins. I have also helped in documenting the plugins design, development and implementation. I mainly re-wrote the document that was written by the GUI plugins group, and compiled all the notes on daemon plugins into a structured document. I was also put in charge of creating the final presentation slides. 

\section*{Tasks and Activities Assigned}

\begin{itemize}
    \item Co-writing plugins documentations.
        \begin{itemize}
            \item Daemon Plugins
                \begin{itemize}
                    \item Description: There were already several prepared notes about daemon plugins that were written by myself. In this sprint, the focus was on getting the plugins integrated and properly documented. The task is to compile the notes into developers comprehensible documents for future reference. 
                    \item Affected files: \texttt{plugin\_group/daemon\_plugins/*}, \texttt{plugin\_group/GUI plugins Doc/*}
                    \item Git commits: \texttt{(pamalite/segp2) efb25d3bb17772440d08e98f53ede927bc1c16cf} \\
                        \texttt{(pamalite/segp2) eee6f84c9a9160d1d17e452b9bc91fb3ecb15676} \\
                        \texttt{(pamalite/segp2) 55801f180e90aa91ddf88f79d2a7762c40fa7d80} 
                    \item Estimated time taken (planned): 2 hours
                    \item Estimated time taken (actual): 2 hours
                \end{itemize}
        \end{itemize}
    \item Strategizing and testing integrated branches.
        \begin{itemize}
            \item Description: Started strategizing with Jon, Fil and Mohammad on how to mitigate merge errors during integration with GIT. Also, I pointed out that we needed a strategy on not relying too much on GIT's merge algorithms as they tend not to merge files properly. Thus, it was suggested to merge files in stages, and each stage is a branch from the master trunk. However, I did not do the merge. I tested the merged stages to help check whether the mergers were properly made. I fixed some errors and committed the corrections back to the respective branches. Some minor merge errors were related to Ticket 66 and missed codes for other tickets which broke the system upon a more in-depth testing was conducted. However, the Unit Test was given up as it was realized that there is not enough time to re-write the extension points codes to cope with Rails' unit test cases. 
            \item Git commits: \texttt{(Jonv/earth/group1\_2) 619b85e55cabe9c959ea128d8e8536d5fc684d96} \\
                \texttt{(Jonv/earth/group1\_2\_earthd\_plugins) ca3c9fda5539cb10881166195df76e99806bcbfb} \\
                \texttt{(Jonv/earth/group1\_2\_earthd\_gui\_plugins) ca3c9fda5539cb10881166195df76e99806bcbfb} \\
                \texttt{(pamalite/earth/group1\_2)  572f62ecb1c8b806d2745ea69a3698af6f1f84c7} \\
                \texttt{(pamalite/earth/group1\_2\_earthd\_plugins) 901bda47ff97d8f4d43a313971653b66b1f15f1a} \\
                \texttt{(pamalite/earth/group1\_2\_earthd\_plugins) 1ea191e6ba50bb2458987dd22bed2eda95d935cf} \\
                \texttt{(pamalite/earth/group1\_2\_earthd\_plugins) f643d7727a9f563ab3c1a04f2039f109f2054788} \\
                \texttt{(pamalite/earth/group1\_2\_earthd\_plugins) fe2aea30ec40e46b4d88ca9960a138f864f81bce} \\
                \texttt{(pamalite/earth/group1\_2\_earthd\_gui\_plugins) d40768224acf690b7fc0395730ffffb1b78c2bff} \\
                \texttt{(pamalite/earth/group1\_2\_earthd\_gui\_plugins) 286490b58f53edb0601672efd33bb3c3757b5d04} \\
                \texttt{(pamalite/earth/group1\_2\_earthd\_gui\_plugins) 901bda47ff97d8f4d43a313971653b66b1f15f1a} \\
                \texttt{(pamalite/earth/group1\_2\_earthd\_gui\_plugins) 1ea191e6ba50bb2458987dd22bed2eda95d935cf} \\
                \texttt{(pamalite/earth/group1\_2\_earthd\_gui\_plugins) f643d7727a9f563ab3c1a04f2039f109f2054788} \\
                \texttt{(pamalite/earth/group1\_2\_earthd\_gui\_plugins) fe2aea30ec40e46b4d88ca9960a138f864f81bce}
            \item Estimated time taken (planned): 4 hours
            \item Estimated time taken (actual): 4 hours
        \end{itemize}
\end{itemize}

\section*{Resource Contributions}

For this milestone, I have estimated that I had spend about 10 hours (including 4 hours meeting time) over the spread of 3 weeks. I have budgeted 45 hours (15 hours per week) for this semester. The other 15 hours were spend on my research project. 

Looking at the hours I felt that I have had a productive sprint, though I am only using a fraction of my budgeted hours. The rest of the hours were used on the research project thesis. However, I observed that my skills on using GIT matured the most in this sprint, due to the fact that I was requested to create a backup repository incase that Jon's repository was mistakenly broken. I have also spend some time to help Jon understand some GIT commands to ease integration and synchronizing with the backup repository. 

\section*{Rooms for Improvement}

\begin{itemize}
   \item Low levels of patience was observed during this sprint. This was because one of the members were not showing interest in helping out with the closing of the project. I lost my patience with the member and had ignored his complaints and ranting completely as I thought this member can be hindrance towards the success of this project, and I focused myself on members who are interested in having a proper project closure. I will try to be more approachable and keep my cool in future projects by trying to have a constructive confrontation with such members. 
\end{itemize}


%\begin{center}
{\large\textbf{\underline{{Xiaodong's Milestone 5 Report}}}}
\end{center}

\section*{Executive Summary}

This is the first milestone since we have regrouped. In this new group, I have been working with other MSE students on plugins. I took about one week to understand the existing concepts about plugin, and discussed with others to understand what needed to be done. I also decomposed the filemonitor into some API files based on the API concepts, after all this, I spent some time to look at the GUI plugins as well.

\section*{Tasks and Activities Assigned}

\begin{itemize}
	\item Decompose filemonitor into seperated API files
	     \begin{itemize}
	        \item Sub-task 1: Looked at the API plugin concepts proposed by other MSE students and the file monitor system.
	           \begin{itemize}
					\item Description: Since I was new to the plugin stuff, it took me some time to look at those stuff, try to understand the whole concept, and had some research on different plugin implementation methods. Also I spent some time to look at the file monitor system in order to find out what can be decomposed to form APIs.
					\item Estimated time taken (planned): 25 hours
					\item Estimated time taken (actual):  20 hours (19/08/08-24/08/08)
				\end{itemize}
			\item Sub-task 2: Decomposed filemonitor into some APIs.
			   \begin{itemize}
					\item Description: We had agreed on the standards of writing API files, and decided that all those API files must be stored under earth\_api folder. I decomposed filemonitor into several API files,(Eapi\_Eta\_printer, Eapi\_file\_monitor, Eapi\_FileMonitor\_benchmark, Eapi\_logger) and reconstructed filemonitor to use methods defined within those API files.
					\item Affected files: 
					\begin{itemize}
					\item  \texttt{file\_monitor.rb}
					\item  \texttt{Eapi\_Eta\_printer}
					\item  \texttt{Eapi\_file\_monitor}
					\item  \texttt{Eapi\_FileMonitor\_benchmark}
					\item  \texttt{Eapi\_logger}
					\end{itemize}
					\item Estimated time taken (planned): 30 hours
					\item Estimated time taken (actual): 22 hours
					(25/08/08-28/08/08)
				\end{itemize}
			\item Sub-task 3: Load filemonitor as a plugin.
			        \begin{itemize}
					\item Description: followed some steps to load filemonitor as a plugin.
					\item Estimated time taken (planned): 5 hours
					\item Estimated time taken (actual): 3 hours (28/08/08-29/08/08)
				\end{itemize}	
	     \end{itemize}
	\item GUI plugin
	     \begin{itemize}
	         \item Sub-task 1: Rails plugin could be applied as GUI plugin for earth.
	            \begin{itemize}
	              \item Description: I had some research on rails plugin, and I think this can be used to write our GUI plugin.
					\item Estimated time taken (planned): 15 hours
					\item Estimated time taken (actual): 12 hours (29/08/08-31/08/08)
				 \end{itemize}
	         \item Sub-task 2: Constructed a simple GUI plugin adding a tab into earth, and display some information under the added tab.
	            \begin{itemize}
				    \item Description: I created a simple rails plugin (user\_usage), included two files one is called user\_usage.rb to include a new method (acts\_as\_add\_tab) to allow uses to create a tab, other one is called usage\_helper used to display information under user\_usage tab.
					\item Affected files: user\_usage plugin folder
					\item Estimated time taken (planned): 6 hour
					\item Estimated time taken (actual): 8 hours (31/08/08-02/09/08)
				 \end{itemize}
	     \end{itemize}
\end{itemize}

\section*{Resource Contributions}

For this milestone, I have estimated that I had spend about 71 hours (including 6 hours meeting time) over the spread of 3 weeks. The first week I was almost spent all the time on reading, because I just moved to this plugin group, I needed to understand what they had done for plugins up to milestone 4. The following two weeks I spent the time on decomposing and GUI plugin stuff.

\section*{Rooms for Improvement}

\begin{itemize}
   \item I think we need better communication between group members. 4 MSE students were assigned the decomposition part, but we seemed to work on that individually. Some API files created by one person might be created by another one as well. In addition, I worked with GUI plugin, and I believe another MSE student was working on that also, so the resources were split. I think we need to focus on one thing, and use our available resources to make it workable. 
\end{itemize}
  

%\begin{center}
{\large\textbf{\underline{{Filimoni's Milestone 5 Report}}}}
\end{center}

\section*{Executive Summary}

For the current milestone (M5), I was reassigned to the Plugin Focus Group. This group comprised all the seven (7) MSE students and 
was tasked with investigating the plugin system for the Earth project. The investigation showed that while the file monitor class 
(file\_monitor.rb) had been designed as a plugin, the actual implementation was tightly coupled to the underlying earth daemon code.
This would make it difficult to expand the features and functions of the Earth project. So the investigation also involved modifying
the Earth daemon code to make it modular and provide a standardised plugin interface.


\section*{Tasks and Activities Assigned}

\begin{itemize}
	\item Examine and Verify Operations of Original Daemon Design and Plugin Support.
		\begin{itemize}
			\item Sub-task 1: Locate and learn about current group progress with the plugin system.
	      	\begin{itemize}
					\item Description: This involves reading the documentation (investigation report) of the plugin system by Group 3 in previous milestones.
					\item Idea/Solution: The existing documents revealed that the original code incorporates the option to implement the file monitor class as a plugin.
					\item Affected files: \texttt{Plugin\_Design\_Notes.pdf}
					\item Git commits: N/A.
					\item Estimated time taken (planned): 10 hours
					\item Estimated time taken (actual): 16 hours
				\end{itemize}
			\item Sub-task 2: Identify all plugin-related designs in codes.
			   \begin{itemize}
					\item Description: While the relevant plugin notes (Sub-task 1) showed the design concepts and explained its implementation, it did not show where these design decisions are implemented in the source codes. As a result, this Sub-task (2) involved identifying the relevant source code files and code snippets in order to attain a better understanding of the original plugin design and implementation. 
					\item Idea/Solution: N/A. 
					\item Affected files: \texttt{lib/earth\_plugins/file\_monitor.rb}, \texttt{script/earthd}
					\item Git commits: \texttt{(ssurfer/earth) 0f0464179d5da4d0d22b3e9345bece9aa02136e4}
					\item Estimated time taken (planned): 20 hours
					\item Estimated time taken (actual): 18 hours
				\end{itemize}
		\end{itemize}
	\item Eliminate the Tight-Coupling of Earth Daemon and File Monitor Plugin
		\begin{itemize}
	   	\item Sub-task 1: Remove hard-coded elements of the file monitor plugin from daemon code.
	      	\begin{itemize}
	         	\item Description: The original form of the Earth daemon involves instantiating the file monitor class and subsequently invoking the (\@file\_monitor) object methods to update the database records of monitored directories. 
					\item Idea/Solution: This particular Sub-task involved changing this process into a generic approach where the daemon simply invokes a general plugin method (main) that is specified in the EarthPlugin superclass and implemented through method override in all the respective plugin subclasses.
					\item Affected files: \texttt{lib/earth\_plugins/file\_monitor.rb}, \texttt{script/earthd}
					\item Git commits: \texttt{(ssurfer/earth) 0f0464179d5da4d0d22b3e9345bece9aa02136e4}
					\item Estimated time taken (planned): 30 hours
					\item Estimated time taken (actual): 24 hours
				\end{itemize}
			\item Sub-task 2: Verify operation of modified daemon and file monitor plugin codes after de-coupling.
	      	\begin{itemize}
					\item Description: Verification that de-coupling of the daemon and the file monitor plugin works as planned.
					\item Idea/Solution:  This Sub-task involved ensuring that in addition to the preservation of the original behaviour of the Earth daemon and file monitor codes implementation, no unnecessary performance degradation occurred as a result of the design modification.  
					\item Affected files: \texttt{lib/earth\_plugins/file\_monitor.rb}, \texttt{script/earthd}
					\item Git commits: \texttt{(ssurfer/earth) 0f0464179d5da4d0d22b3e9345bece9aa02136e4}
					\item Estimated time taken (planned): 10 hour
					\item Estimated time taken (actual): 16 hours
				\end{itemize}
		\end{itemize}
	\item Identify Effective Plugin Design
		\begin{itemize}
	   	\item Description: Examined ways to generically implement or load plugins from the daemon. 
	      \item Idea/Solution: The most straight-forward way to implement the plugins was to simply load all existing plugins from the database. Conceptually, this idea also implies that the plugin itself has to have extra-functional features and logic blocks that determines runtime invocation and/or deactivation. The downside of this simplistic implementation means a signficant overhead on the plugin class, which would probably discourage the development of additional feature plugins. A further downside is that any interested plugin developer would need to access and study the daemon code in order to determine how his/her plugin will 'hookup' onto the daemon.
	      \item Affected files: \texttt{lib/earth\_plugins/file\_monitor.rb}, \texttt{script/earthd}
	      \item Git commits: \texttt{(ssurfer/earth) 0f0464179d5da4d0d22b3e9345bece9aa02136e4}
	      \item Estimated time taken (planned): 20 hours
	      \item Estimated time take (actual): 12 hours
		\end{itemize}
\end{itemize}

\section*{Resource Contributions}

After contributing about 86 hours over the course of 3 weeks during Milestone 5, I believe that more progress could have been achieved
if the organisation of existing documents about the plugin system had been up-to-date and less cryptic than what it was at the commencement
date of this current development sprint(M5). As a result, a significant portion of this 86 hours was spent on verifying the accuracy and
correctness of the documentation. \\

However, I believe that this situation was only applicable to those of us who were not part of Subgroup 3 during the previous development sprints.
Subgroup 1 (my former group) had been exposed to every aspect of the Earth project except the plugin system in previous sprints. This might help explain
why a few of us were not as well-versed with the available documentation at the beginning of this sprint. This situation was rectified with
the appropriate updates of the plugin system documentation and a special briefing session conducted by Ken.


\section*{Room for Improvement}

\begin{itemize}
   \item Better planning: Having coarse-grained Sub-tasks provides room for unproductive investigations that resulted in wasted resources.
      \begin{itemize}
         \item Plan: Refine subtasks further into 3-4 hours tasks for better monitoring and improved development focus. 
      \end{itemize}
\end{itemize}


%\begin{center}
{\large\textbf{\underline{{Kun Zhou's Milestone 5 Report}}}}
\end{center}

\section*{Executive Summary}

After previous milestone, the main tasks were changed to plugin system and integrated testing. So the MSE students were grouped together to focus on plugin system. Based on the progress of last Gourp3 in milestone 4, we were assigned different task. I and the other 3 students should finish the split-up of existed file monitor and some research on plugin for GUI. The basic API for plugin has been created. But the API system should cost much more time to be perfect in the future. For GUI plugin, I and Jian Huang made a different plugin management system compared with Rails Plugin. The basic function can be used for plug some pages in the Controller side. Therefore, a lot of blocks need to be broken.

\section*{Tasks and Activities Assigned}

\begin{itemize}
	\item Daemon API and new File Monitor
	     \begin{itemize}
	        \item Sub-task 1: Analyse the File Monitor to find a better way to split it into API
	           \begin{itemize}
					\item This is the most first time for me to analyse the file monitor. I have not touched both of the earthd and file monitor before. That is a big challenge for me to understand the whole system and gain the relationship between different parts of file monitor. Unfortunately, more than 20 hours have been spent on it, including searching information from books and internet. 
					\item Estimated time taken (planned): 18 hours
					\item Estimated time taken (actual): 27 hours
				\end{itemize}
			\item Sub-task 2: Create a new file monitor treated as a plugin
			   \begin{itemize}
					\item Description: Honestly, it took much less time than i expected. The whole process is simple. Only two parts of file monitor can be separated into two classes. Obviously, the testing went well.
					\item Affected files: \texttt{plugin\_stuff.tar.gz} 
					\item Google Group: \begin{verbatim}http://segp2.googlegroups.com/web/plugin_stuff.tar.gz?hl=en&gda=fD840
					EUAAAAZYjv4r0sNEon_mhrM-rXNDDviO9ZU7gSmNAhfKCcIXE4EF7-alpc9UU2N5vbkz
					bJuoARl3CxVwlYIPn-dXdeGu1iLHeqhw4ZZRj3RjJ_-A\end{verbatim}
					\item Estimated time taken (planned): 20 hours
					\item Estimated time taken (actual): 10 hours
				\end{itemize}
	     \end{itemize}
	\item Run new plugin from database
	     \begin{itemize}
	         \item Sub-task1: sign and install the file monitor into database.
	            \begin{itemize}
							\item Description: It is easy for me by following the guider from ken. 
							\item Estimated time taken (planned): 5 hours
							\item Estimated time taken (actual): 5 hours
							\end{itemize}
	         \item Sub-task 2: Call file monitor from database
	            \begin{itemize}
							\item Description: Unfortunately, this is the biggest problem i have ever meet. The encode problem! I almost spent the whole weekend on this weird problem. The verification is the key for security of plugin in earth system. When the lines of plugin file is less than around 300, the '\textbackslash n 'would be changed to '\textbackslash 012'. At first I can't understand the reason of the problem. After the discuss in group meeting, ken got a solution using Base64 to encode the file first in order to avoid the appearance of \'\\n\'. After testing, it is proved as a best solution so fa. 
							\item Affected files: \texttt{plugin\_stuff.tar.gz} 
							\item Google Group: \begin{verbatim}http://segp2.googlegroups.com/web/plugin_stuff.tar.gz?hl=en&gda=fD840
							EUAAAAZYjv4r0sNEon_mhrM-rXNDDviO9ZU7gSmNAhfKCcIXE4EF7-alpc9UU2N5vbkz
							IbJuoARl3CxVwlYIPn-dXdeGu1iLHeqhw4ZZRj3RjJ_-A\end{verbatim}
							\item Estimated time taken (planned): 18 hours
							\item Estimated time taken (actual): 26 hours
							 \end{itemize}
	     \end{itemize}
	\item Help Jian Huang to realize his idea on GUI plugin
	     \begin{itemize}
	         \item Description: This task was self-assigned.After the meeting with supervisor, I think Jian Huang's idea it possible to solve the plugin problem for GUI. Therefore, I decided to help him and work together for a couple of days on this aim. Now it can be presented but a lot of features are still needed to be done. The created plugin system influence the original earth system huge. So it did not uploaded into the git repository. In the next milestone, the group would discuss the functionality of it and decided whether it can be used for the new GUI plugin system.
	         \item Estimated time taken (actual): 17 hour
	     \end{itemize}
	\item Low level separation of API
	     \begin{itemize}
	         \item Description: This task was self-assigned for make API basically in order to make the new plugin can use it easily. But it haven't started. I would finish it in the next milestone as soon as possible
	         \item Estimated time taken (actual): 8 hours
	     \end{itemize}
\end{itemize}

\section*{Resource Contributions}

In this Milestone, I spent too much time on the analysis of file monitor and solution of the strange problem, almost 50 hours. Most of the other hours were mainly spending on working with my group member. But after this milestone, I have a whole structure of earthd including file monitor and understand the way it works.

\section*{Rooms for Improvement}


The first problem in this milestone for me is my lacking of experience of development about earth daemon. Fortunately, i have passed and got a lot of skills. It should be improved in the future. Additionally, I cost too much time on solution of a particular problem and tried to gain it individually. I should have sent it to the whole group and discusses with the others. It should be a better way to conquer the wall of difficulties.


%\markright{Mohammad Bamogaddam (1144332)}
\pagestyle{headings}

\begin{center}
{\large\textbf{\underline{{Mohammad's Milestone 6 Report}}}}
\end{center}

\section*{Executive Summary}
This report records my activities related to Earth project development for Milestone 6 period from 7-September-2008 to 12-October-2008.\\
In this sprint,I have finished the following main tasks: review the earth daemon plugin framework, prepare a document about the earth daemon plugin framework design, check the earthd status script and review the plugin installation script.      

\section*{Tasks and Activities Assigned}

\begin{itemize}
    \item Review the earth daemon plugin framework.
        \begin{itemize}
            \item Sub Task 1:
                \begin{itemize}
                    \item Description: reviewing the metadata plugins which was created in the previews milestone.
                    \item Affected Files: \texttt{rsp\_add\_file\_metadata.rb, rsp-add-file-metadata.rb, rsp\_delete\_file\_metadata.rb, rsp-delete-file-metadata.rb} and \texttt{rsp\_delete\_under\_dir\_metadata.rb, rsp-delete-under-dir-metadata.rb}
                    \item Git Commits: check the commits on the next sub-task.
                \end{itemize}
            \item Sub Task 2: Creating a metadata API which can be used for all metadata plugins in future
                \begin{itemize}
                    \item Description: I create an API contains common metadata functionality like: saving, deleting, searching, etc. All metadata plugins should use this API and not use earth models directly. 
                    \item Affected files: \texttt{metadata-api.rb}
                    \item Git Commits: These Commits includes the previous sub-task commits. 
\texttt{mfbDev/earth: 8a1b6113590b28e, 3742a36327dde, 3742a36327dde, 48da91b6f4d, 48da91b6f4d, a4b173b98b7a5, a4b173b98b7a5, 946ba2c55f83, 946ba2c55f83, b715eee7d38, b715eee7d38, cb98d73ac84, cb98d73ac84}
                \end{itemize}
            \item Sub Task 3: Testing the metadata API
                \begin{itemize}
                    \item Description: testing the metadata API with the plugins  
                    \item Affected Files: N/A
                    \item Git Commits: N/A
                \end{itemize}
        \end{itemize} 
\newpage     
    \item Prepare a document about the earth daemon plugin framework design
        \begin{itemize}
            \item Description: I created a document that includes details about the earth daemon plugins framework design. This document includes the following main sections: Overall structure, extension points, APIs, plugins loading, simple plugin example and steps to create a plugin. So, this document should be a guide or any one who want to create a new earth daemon plugin. 
            \item Affected Files: \texttt{earth\_plugin\_framework.pdf, earth-plugin-framework.pdf}
            \item Git Commits: \texttt{pamalite/segp2: 68f83a173bdc, 68f83a173bdc}
        \end{itemize}
    \item Review the plugin installation script 
        \begin{itemize}
            \item Description: Ida has been working on improving the plugin installation script so it will work with the new framework design (i.e. extension points). My task was reviewing her work and testing it. 
            \item Affected Files: \texttt{plugin\_manager.rb, plugin-manager.rb} and \texttt{uninstall\_plugin, uninstall-plugin}
            \item Git Commits: 
\texttt{mfbDev/earth: 465d98fa, 465d98fa, dd4f1ed5a7, dd4f1ed5a7, 0706e307042, 0706e307042, 177e3e4924, 177e3e4924} 
        \end{itemize}
    \item Check the earthd status script
        \begin{itemize}
            \item Description: We noticed a problem with the (status) script for earth daemon. When we install File-Monitor as a plugin, this script does not work anymore. This took lots of time to figure out the problem. At the beginning, I thought it is something related to Unix sockets since the system hangs on a (socket.receive) method. I did not find any documentation for this method in Rails API or in the Internet!!! But after several attempts, I figure out the problem. Earth daemon is reading the (status) from an instance variable from the File-Monitor. That's why when we install File-Monitor as a plugin, it does not work anymore. The solution for this could be: 
                \begin{enumerate}
                    \item rewrite the status script
                    \item use File-Monitor as a normal file because it is the heart of earth daemon
                    \item create an extension point for reading the status (I am not sure if this is feasible)
                \end{enumerate}
This task is pushed to be finished at the beginning of next milestone

            \item Affected Files: \texttt{earthd}
            \item Git Commits: N/A
        \end{itemize}
    \end{itemize}
\newpage
\section*{Resource Contributions}
I estimated that I will spent about 60 hours for this milestone. I have spent about 50 hours including meeting times. 


\section*{Milestone Observations}
I am happy with the outcome from this milestone from our group. Every one is finishing his assigned tasks and we made a good progress. One thing, we need to give more attention to the documentation and code comments on the next milestone.



%\begin{center}
{\large\textbf{\underline{{Qing Yang Milestone 5 Report}}}}
\end{center}

\section*{Executive Summary}
For this milestone, all the MSE students are regrouped together focusing on the plugin task.I was assigned to the decomposing subgroup. Due to it was my first time to get in touch with the plugin related work,I spent the first week of this sprint in understanding the structure of earth plugin and how earth plugin worked on earth.
In the second week ,I thought about how to decomposing the existing plugin filemonitor into apis and then I self-assigned myself to the work of security part of plugin,including creating certificate and keys, sign plugin with the key and put everything into database.However,when I was working on it, I found a bug,but I was not sure whether the bug was from, the ruby::openssl or ruby postgres driver.Then Ken decided to solve the problem by using string to represent the signature rather than the binary.I also spent a lot of time on troubleshooting, The main trouble was that my earth running on Aptenna and earth running in console had a conflict.

\section*{Tasks and Activities Assigned}

\begin{itemize}
	\item Create plugin APIs in term of Ken's API standard.
	     \begin{itemize}
	        \item Sub-task 1: Have a research on earth plugin.
	           \begin{itemize}
				        \item Description:The research includes what is earth plugin , the relationships among filemonitor and other parts of earth, namely ,how they call and work with one another and how plugins work on earth,rather how to create a plugin and how to install a plugin.
                                        \item Affected files: \\
					 \texttt{earth/lib/earth\_plugins\\file\_file\_monitor.rb}\\
                                        \texttt{earth/lib/earth\_plugin\_interface\\plugin\_manager.rb}\\
                                        \texttt{earth/script/earthd}\\
                                        \texttt{earth/script/create\_cert}\\
                                        \texttt{earth/script/sign\_plugin}\\
                                        \texttt{earth/script/install\_plugin} \\
					\item Estimated time taken (planned):20 hours\\
					\item Estimated time taken  (actual): 21 hours\\
				\end{itemize}
                                \end{itemize}  

                    \begin{itemize}
			 \item  Sub-task 2:Working with group members to decompose the plugin APIs .
                         \begin{itemize}
					\item Description: Discussing with group members and In terms of Ken's   API standard, we decomposed the existing plugin 
                                        file\_monitor.rb into APIs.
                                        \item Affected files: \\
                                             file\_monitor.rb\\
                                             api\_eta\_printer.rb\\
                                             api\_file\_monitor.rb\\
					\item Estimated time taken (planned):10 hours
					\item Estimated time taken (actual): 10 hours
		       \end{itemize}              
                       \end{itemize}     
           \end{itemize}

 \begin{itemize}
\item Fixing the conflict between earth running on Aptena and earth running in console.
	     \begin{itemize}
	         \item Sub-task 1: Find out the reason why I cannot run earth in console by using the command 
                  ../script/earthd start
	            \begin{itemize}
	              \item Description:I found the socket was occupied by Aptena ,so that earth cannot work on        console. 		
			   	       \item Affected files:\\
                                      \item /tmp/earthd.sock\\
				      \item Estimated time taken (planned):6 hours
				      \item Estimated time taken (actual): 10hours
	                  \end{itemize}
                          \end{itemize}
               \begin{itemize}
	         \item Sub-task 2: Uninstall the Aptena Studio plugin on Eclipse
	            \begin{itemize}
				  \item Description: At first I did not know how to uninstall it.However, it                        was done finally.
			          \item Estimated time taken (planned):5hour
				  \item Estimated time taken (actual): 7hours
                   \end{itemize}
                   \end{itemize}
  \end{itemize}

\begin{itemize}
\item Work on the security part.
	     \begin{itemize}
              \item Sub-task 1: Learn how to sign and install a plugin.
                \begin{itemize}
	         \item Description: This task included creating certificate,private and public key pair, sign the code of plugin with the private key ,and verify the code and it's signature with the public key in certifucate, when installing a plugin. I spent a lot of time to read the plugin\_manager and earthd and learned how to use openssl to do every thing, and felt comfortable to write all the security related functionalities by myself.It indeed took me a lot of time.But When I read Ken's plugin notes,I knew that all I had done were just waste of time.RSP have already finished everything for us.What we need to do was typing create\_cert, sign\_plugin and install\_plugin in command line. 
	         \item Affected files: \\
                                       \texttt{earth/lib/earth\_plugin\_interface/plugin\_manager.rb}\\
                                        \texttt{earth/script/earthd}\\
                                        \texttt{earth/script/create\_cert}\\
                                        \texttt{earth/script/sign\_plugin}\\
                                        \texttt{earth/script/install\_plugin}\\ 
	      
	          \item Estimated time taken (planed): 30 hour
                  \item Estimated time taken (actual): 20 hour
	     \end{itemize}
             \end{itemize}

     \begin{itemize}
            \item Sub-task 2: Put the code,signature, and plugin's name and version into database. And install                  file monitor as a plugin.
	            \begin{itemize}
				    \item Description: Both Keane and I found that if we put the original file\_monitor into the database, when we ran ./script/install\_plugin the signature can be converted properly which was the same as the code in file\_monitor.rb. But once we broke the it into APIs and call the methods that we need from the new created file\_monitor file, everything changed. The weired problem is all the '$\setminus$n' was convered into '$\setminus$012',so that the code and signature pair could not pass the verification. I spent huge amount of time in testing.I compared the original file\_monitor with the new created file\_monitor and try to find out the source of the problem. I broke down the original file\_monitor little by little,namely the size of file\_monitor was becoming smaller and smaller. At first it could work properly, all the '$\setminus$n' signals were converted in the right way. But when I keeping making its size smaller. The file\_monitor was broken by sudden. Therefore, I was sure that this problem depanded on the size of the file. I considered that it was a bug 
                                   However,I was not sure where the bug was from, openssl or postgres driver. And finally Ken solved this problem by using string into code and signature in database, instead of binary.
					\item Estimated time taken (planned):10hour
					\item Estimated time taken (actual): 25hours
		     \end{itemize}
                     \end{itemize}
  \end{itemize}

\section*{Resource Contributions}

For this milestone, I have estimated that I had spend about 95 hours (including 4 hours meeting time) over the spread of 3 weeks.

\section*{Rooms for Improvement}

\begin{itemize}
   \item We have a big problem in the time and difficulty estimation. And tasks did not assigned to  each team        member properly.I found that at least half of the  team members did not feel comfortable to work in the       group. 
      \begin{itemize}
         \item Review more on past milestones, and perform reality checks during planning.
      \end{itemize}

     \begin{itemize}
        \item  Better communication with other members.
          Team members should trust one another,and be more helpful.
          Estimation efficiency should be improved.       
     \end{itemize}
\end{itemize}


%
\begin{center}
{\large\textbf{\underline{{Huang's Milestone 4 Report}}}}
\end{center}

\section*{Executive Summary}
For this milestone, all the MSE students are regrouped together focusing on the plugin task.I was assigned to the decomposing subgroup. Due to it was my first time to get in touch with the plugin related work,I spent the first week of this sprint in understanding the structure of earth plugin and how earth plugin worked on earth.
In the second week ,I thought about how to decomposing the existing plugin filemonitor into apis and then I self-assigned myself to the work of security part of plugin,including creating certificate and keys, sign plugin with the key and put everything into database.However,when I was working on it, I found a bug,but I was not sure whether the bug was from, the ruby::openssl or ruby postgres driver.Then Ken decided to solve the problem by using string to represent the signature rather than the binary.I also spent a lot of time on troubleshooting, The main trouble was that my earth running on Aptenna and earth running in console had a conflict.

\section*{Tasks and Activities Assigned}

\begin{itemize}
	\item Create plugin APIs in term of Ken's API standard.
	     \begin{itemize}
			 \item  Sub-task: Working with group members to decompose the plugin APIs .
                         \begin{itemize}
					\item Description: Discussing with group members and In terms of Ken's   API standard, we decomposed the existing plugin 
                                        file\_monitor.rb into APIs.
                                        \item Affected files: \\
                                             file\_monitor.rb\\
                                             api\_eta\_printer.rb\\
                                             api\_file\_monitor.rb\\
					\item Estimated time taken (planned):5 hours
					\item Estimated time taken (actual): 5 hours
		       \end{itemize}              
                       \end{itemize}     
           \end{itemize}

 \begin{itemize}
\item research GUI plugin.
	     \begin{itemize}
	         \item Sub-task 1: understand how the rails plugin works
	            \begin{itemize}
	              \item Description: read relative books: the rail way, ruby on rail, and so on. After that, briefly illustrate how the rails plugin works, and what's the pro and con, if we use it. Think about my own way to improve it
			   	       \item Affected files:\\
                                      \item /tmp/earthd.sock\\
				      \item Estimated time taken (planned):40 hours
				      \item Estimated time taken (actual): 50hours
	                  \end{itemize}
                          \end{itemize}
               \begin{itemize}
	         \item Sub-task 2: build my way to create Gui plugin instead of rails plugin. In my way, I do not need to change earth coding and Gui plugin can be run in different ways, which will depends on my plugin management.
	            \begin{itemize}
				  \item Description: share my idea with group member, discuss it
			          \item Estimated time taken (planned):20 hours
				  \item Estimated time taken (actual): 20 hours
                   \end{itemize}
                   \end{itemize}
  \end{itemize}

\begin{itemize}
\item change Ticket 66 which ken has done last semester into a Gui plugin.
	     \begin{itemize}
              \item Sub-task 1: .rebuild ticket 66 as a Gui plugin
                \begin{itemize}
	         \item Description: The reason to choose this ticket is that it involves three parts: view, controller and module. Meanwhile, the most difficult part is change / extend controller, not create a new one.
	          \item Estimated time taken (planed): 20 hour
                  \item Estimated time taken (actual): 30 hour
	     \end{itemize}
             \end{itemize}

     \begin{itemize}
            \item Sub-task 2: Test Gui plugin, change some parts of coding again, because what I've got the version of earth has been changed many times, and some parts of coding are not commented.
	            \begin{itemize}
				    \item Description: Thank keane for doing my a favour.
					\item Estimated time taken (planned):30hour
					\item Estimated time taken (actual): 40hours
		     \end{itemize}
                     \end{itemize}
  \end{itemize}

\section*{Resource Contributions}

I'm pretty tired in this milestone, because of spending too much time on coding and testing, which means I'm not good at programming.

\section*{Rooms for Improvement}

\begin{itemize}
   \item We have a big problem in the time and difficulty estimation. And tasks did not assigned to  each team        member properly.I found that at least half of the  team members did not feel comfortable to work in the       group. 
      \begin{itemize}
         \item Review more on past milestones, and perform reality checks during planning.
      \end{itemize}

     \begin{itemize}
        \item  Better communication with other members.
          Team members should trust one another,and be more helpful.
          Estimation efficiency should be improved.       
     \end{itemize}
\end{itemize}


\end{document}  